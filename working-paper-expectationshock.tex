\documentclass[12pt]{article}
\pdfminorversion=5 
\pdfcompresslevel=9
\pdfobjcompresslevel=2
\usepackage{amssymb}
\usepackage{amsmath}
\usepackage{geometry}
\usepackage{graphicx}
\usepackage{color}
\usepackage[T1]{fontenc}
\usepackage[utf8]{inputenc}
\usepackage{mathtools}
\usepackage{lmodern}
\usepackage{caption}
\usepackage[table]{xcolor}
\usepackage{subfigure}
\usepackage{bigints}
\usepackage{bbm}
\usepackage{xr}
\usepackage{lipsum}
\usepackage{pdfpages}
\usepackage{xcolor}
\usepackage{mathrsfs}
\definecolor{redd}{rgb}{0.8, 0.1, 0.1}
\definecolor{navyblue}{rgb}{0, 0.6, 0.2}
\definecolor{amaranth}{rgb}{0.9, 0.17, 0.31}
\definecolor{alizarin}{rgb}{0.82, 0.1, 0.26}
\definecolor{bostonuniversityred}{rgb}{0.8, 0.0, 0.0}
\definecolor{brickred}{rgb}{0.8, 0.25, 0.33}
\definecolor{cornellred}{rgb}{0.7, 0.11, 0.11}
\usepackage[colorlinks,linkcolor=navyblue,urlcolor=blue,citecolor=navyblue]{hyperref}
\newcommand{\navy}[1]{\textcolor{blue}{\bf #1}}
\newcommand{\navymth}[1]{\textcolor{blue}{#1}}
\newcommand{\red}[1]{\textcolor{red}{#1}}
\usepackage{subfigure}
\usepackage[authoryear,round]{natbib}
\usepackage{sectsty}
\usepackage{multirow}
\usepackage{rotating}
\usepackage[]{morefloats}
\usepackage{booktabs}
\usepackage{float}
%\usepackage[runin]{abstract}
%\abslabeldelim{\;}
\usepackage{fancyhdr}
\usepackage{amsmath}
\usepackage{threeparttable}
%\usepackage{mathptmx}
%\usepackage{newtxmath}
%\usepackage{times}
\usepackage{tikz}
\usetikzlibrary{shapes.geometric, arrows}
\usepackage{rotating}
\UseRawInputEncoding
\usepackage{tabularx}
\usepackage{tabulary}
\usepackage[newcommands]{ragged2e}
\usepackage{tabularx}
\usepackage{adjustbox}
\usepackage{setspace} 
\doublespacing
\usepackage{mathpazo}
\usepackage{etoolbox}

\setcounter{MaxMatrixCols}{12}





\geometry{left=1.0in,right=1.0in,top=1.0in,bottom=1.0in}

\parskip5pt
\parindent15pt
\renewcommand{\baselinestretch}{1.1}

\newcommand*{\theorembreak}{\usebeamertemplate{theorem end}\framebreak\usebeamertemplate{theorem begin}}

\newcommand{\newtopic}[1]{\textcolor{Green}{\Large \bf #1}}

\newcommand\Taccount[3][3cm]%
{{\renewcommand\arraystretch{1.3}%
		\begin{adjustbox}{width=0.4\textwidth}
		\begin{tabular}[t]{@{}p{#1}|p{#1}@{}}
			\multicolumn{2}{@{}c@{}}{#2}\\
			\hline
			\hline
			#3
		\end{tabular}%
		\end{adjustbox}
	
}}


\definecolor{pale}{RGB}{235, 235, 235}
\definecolor{pale2}{RGB}{175,238,238}
\definecolor{turquois4}{RGB}{0,134,139}

% Typesetting code
\definecolor{bg}{rgb}{0.95,0.95,0.95}
\usepackage{minted}
\usemintedstyle{friendly}
\newminted{python}{mathescape,frame=lines,framesep=4mm,bgcolor=bg}
\newminted{ipython}{mathescape,frame=lines,framesep=4mm,bgcolor=bg}
\newminted{julia}{mathescape,frame=lines,framesep=4mm,bgcolor=bg}
\newminted{c}{mathescape,linenos=true}
\newminted{r}{mathescape,  frame=none, baselinestretch=1, framesep=2mm}
\renewcommand{\theFancyVerbLine}{\sffamily
	\textcolor[rgb]{0.5,0.5,1.0}{\scriptsize {\arabic{FancyVerbLine}}}}


\usepackage{stmaryrd}

\newcommand{\Fact}{\textcolor{Brown}{\bf Fact. }}
\newcommand{\Facts}{\textcolor{Brown}{\bf Facts }}
\newcommand{\keya}{\textcolor{turquois4}{\bf Key Idea. }}
\newcommand{\Factnodot}{\textcolor{Brown}{\bf Fact }}
\newcommand{\Eg}{\textcolor{ForestGreen}{Example. }}
\newcommand{\Egs}{\textcolor{ForestGreen}{Examples. }}
\newcommand{\Ex}{{\bf Ex. }}
\newcommand{\Thm}{\textcolor{Brown}{\bf Theorem. }}
\newcommand{\Prf}{\textcolor{turquois4}{\bf Proof.}}
\newcommand{\Ass}{\textcolor{turquois4}{\bf Assumption.}} 
\newcommand{\Lem}{\textcolor{Brown}{\bf Lemma. }}

%source code 



% cali
\usepackage{mathrsfs}
\usepackage{bbm}
\usepackage{subfigure}

\newcommand{\argmax}{\operatornamewithlimits{argmax}}
\newcommand{\argmin}{\operatornamewithlimits{argmin}}

\newcommand\T{{\mathpalette\raiseT\intercal}}
\newcommand\raiseT[2]{\raisebox{0.25ex}{$#1#2$}}

\DeclareMathOperator{\cl}{cl}
%\DeclareMathOperator{\argmax}{argmax}
\DeclareMathOperator{\interior}{int}
\DeclareMathOperator{\Prob}{Prob}
\DeclareMathOperator{\kernel}{ker}
\DeclareMathOperator{\diag}{diag}
\DeclareMathOperator{\sgn}{sgn}
\DeclareMathOperator{\determinant}{det}
\DeclareMathOperator{\trace}{trace}
\DeclareMathOperator{\Span}{span}
\DeclareMathOperator{\rank}{rank}
\DeclareMathOperator{\cov}{cov}
\DeclareMathOperator{\corr}{corr}
\DeclareMathOperator{\range}{rng}
\DeclareMathOperator{\var}{var}
\DeclareMathOperator{\mse}{mse}
\DeclareMathOperator{\se}{se}
\DeclareMathOperator{\row}{row}
\DeclareMathOperator{\col}{col}
\DeclareMathOperator{\dimension}{dim}
\DeclareMathOperator{\fracpart}{frac}
\DeclareMathOperator{\proj}{proj}
\DeclareMathOperator{\colspace}{colspace}

\providecommand{\inner}[1]{\left\langle{#1}\right\rangle}

% mics short cuts and symbols
% mics short cuts and symbols
\newcommand{\st}{\ensuremath{\ \mathrm{s.t.}\ }}
\newcommand{\setntn}[2]{ \{ #1 : #2 \} }
\newcommand{\cf}[1]{ \lstinline|#1| }
\newcommand{\otms}[1]{ \leftidx{^\circ}{#1}}

\newcommand{\fore}{\therefore \quad}
\newcommand{\tod}{\stackrel { d } {\to} }
\newcommand{\tow}{\stackrel { w } {\to} }
\newcommand{\toprob}{\stackrel { p } {\to} }
\newcommand{\toms}{\stackrel { ms } {\to} }
\newcommand{\eqdist}{\stackrel {\textrm{ \scriptsize{d} }} {=} }
\newcommand{\iidsim}{\stackrel {\textrm{ {\sc iid }}} {\sim} }
\newcommand{\1}{\mathbbm 1}
\newcommand{\dee}{\,{\rm d}}
\newcommand{\given}{\, | \,}
\newcommand{\la}{\langle}
\newcommand{\ra}{\rangle}

\renewcommand{\rho}{\varrho}

\newcommand{\htau}{ \hat \tau }
\newcommand{\hgamma}{ \hat \gamma }

\newcommand{\boldx}{ {\mathbf x} }
\newcommand{\boldu}{ {\mathbf u} }
\newcommand{\boldv}{ {\mathbf v} }
\newcommand{\boldw}{ {\mathbf w} }
\newcommand{\boldy}{ {\mathbf y} }
\newcommand{\boldb}{ {\mathbf b} }
\newcommand{\bolda}{ {\mathbf a} }
\newcommand{\boldc}{ {\mathbf c} }
\newcommand{\boldi}{ {\mathbf i} }
\newcommand{\bolde}{ {\mathbf e} }
\newcommand{\boldp}{ {\mathbf p} }
\newcommand{\boldq}{ {\mathbf q} }
\newcommand{\bolds}{ {\mathbf s} }
\newcommand{\boldt}{ {\mathbf t} }
\newcommand{\boldz}{ {\mathbf z} }

\newcommand{\boldzero}{ {\mathbf 0} }
\newcommand{\boldone}{ {\mathbf 1} }

\newcommand{\boldalpha}{ {\boldsymbol \alpha} }
\newcommand{\boldbeta}{ {\boldsymbol \beta} }
\newcommand{\boldgamma}{ {\boldsymbol \gamma} }
\newcommand{\boldtheta}{ {\boldsymbol \theta} }
\newcommand{\boldxi}{ {\boldsymbol \xi} }
\newcommand{\boldtau}{ {\boldsymbol \tau} }
\newcommand{\boldepsilon}{ {\boldsymbol \epsilon} }
\newcommand{\boldmu}{ {\boldsymbol \mu} }
\newcommand{\boldSigma}{ {\boldsymbol \Sigma} }
\newcommand{\boldOmega}{ {\boldsymbol \Omega} }
\newcommand{\boldPhi}{ {\boldsymbol \Phi} }
\newcommand{\boldLambda}{ {\boldsymbol \Lambda} }
\newcommand{\boldphi}{ {\boldsymbol \phi} }

\newcommand{\Sigmax}{ {\boldsymbol \Sigma_{\boldx}}}
\newcommand{\Sigmau}{ {\boldsymbol \Sigma_{\boldu}}}
\newcommand{\Sigmaxinv}{ {\boldsymbol \Sigma_{\boldx}^{-1}}}
\newcommand{\Sigmav}{ {\boldsymbol \Sigma_{\boldv \boldv}}}

\newcommand{\hboldx}{ \hat {\mathbf x} }
\newcommand{\hboldy}{ \hat {\mathbf y} }
\newcommand{\hboldb}{ \hat {\mathbf b} }
\newcommand{\hboldu}{ \hat {\mathbf u} }
\newcommand{\hboldtheta}{ \hat {\boldsymbol \theta} }
\newcommand{\hboldtau}{ \hat {\boldsymbol \tau} }
\newcommand{\hboldmu}{ \hat {\boldsymbol \mu} }
\newcommand{\hboldbeta}{ \hat {\boldsymbol \beta} }
\newcommand{\hboldgamma}{ \hat {\boldsymbol \gamma} }
\newcommand{\hboldSigma}{ \hat {\boldsymbol \Sigma} }

\newcommand{\boldA}{\mathbf A}
\newcommand{\boldB}{\mathbf B}
\newcommand{\boldC}{\mathbf C}
\newcommand{\boldD}{\mathbf D}
\newcommand{\boldI}{\mathbf I}
\newcommand{\boldL}{\mathbf L}
\newcommand{\boldM}{\mathbf M}
\newcommand{\boldP}{\mathbf P}
\newcommand{\boldQ}{\mathbf Q}
\newcommand{\boldR}{\mathbf R}
\newcommand{\boldX}{\mathbf X}
\newcommand{\boldU}{\mathbf U}
\newcommand{\boldV}{\mathbf V}
\newcommand{\boldW}{\mathbf W}
\newcommand{\boldY}{\mathbf Y}
\newcommand{\boldZ}{\mathbf Z}

\newcommand{\bSigmaX}{ {\boldsymbol \Sigma_{\hboldbeta}} }
\newcommand{\hbSigmaX}{ \mathbf{\hat \Sigma_{\hboldbeta}} }

\newcommand{\RR}{\mathbbm R}
\newcommand{\CC}{\mathbbm C}
\newcommand{\NN}{\mathbbm N}
\newcommand{\PP}{\mathbbm P}
\newcommand{\EE}{\mathbbm E \nobreak\hspace{.1em}}
\newcommand{\EEP}{\mathbbm E_P \nobreak\hspace{.1em}}
\newcommand{\ZZ}{\mathbbm Z}
\newcommand{\QQ}{\mathbbm Q}


\newcommand{\XX}{\mathcal X}

\newcommand{\aA}{\mathcal A}
\newcommand{\fF}{\mathscr F}
\newcommand{\bB}{\mathscr B}
\newcommand{\iI}{\mathscr I}
\newcommand{\rR}{\mathscr R}
\newcommand{\dD}{\mathcal D}
\newcommand{\lL}{\mathcal L}
\newcommand{\llL}{\mathcal{H}_{\ell}}
\newcommand{\gG}{\mathcal G}
\newcommand{\hH}{\mathcal H}
\newcommand{\nN}{\textrm{\sc n}}
\newcommand{\lN}{\textrm{\sc ln}}
\newcommand{\pP}{\mathscr P}
\newcommand{\qQ}{\mathscr Q}
\newcommand{\xX}{\mathcal X}

\newcommand{\ddD}{\mathscr D}


\newcommand{\R}{{\texttt R}}
\newcommand{\risk}{\mathcal R}
\newcommand{\Remp}{R_{{\rm emp}}}

\newcommand*\diff{\mathop{}\!\mathrm{d}}
\newcommand{\ess}{ \textrm{{\sc ess}} }
\newcommand{\tss}{ \textrm{{\sc tss}} }
\newcommand{\rss}{ \textrm{{\sc rss}} }
\newcommand{\rssr}{ \textrm{{\sc rssr}} }
\newcommand{\ussr}{ \textrm{{\sc ussr}} }
\newcommand{\zdata}{\mathbf{z}_{\mathcal D}}
\newcommand{\Pdata}{P_{\mathcal D}}
\newcommand{\Pdatatheta}{P^{\mathcal D}_{\theta}}
\newcommand{\Zdata}{Z_{\mathcal D}}


\newcommand{\e}[1]{\mathbbm{E}[{#1}]}
\newcommand{\p}[1]{\mathbbm{P}({#1})}

%\theoremstyle{plain}
%\newtheorem{axiom}{Axiom}[section]
%\newtheorem{theorem}{Theorem}[section]
%\newtheorem{corollary}{Corollary}[section]
%\newtheorem{lemma}{Lemma}[section]
%\newtheorem{proposition}{Proposition}[section]
%
%\theoremstyle{definition}
%\newtheorem{definition}{Definition}[section]
%\newtheorem{example}{Example}[section]
%\newtheorem{remark}{Remark}[section]
%\newtheorem{notation}{Notation}[section]
%\newtheorem{assumption}{Assumption}[section]
%\newtheorem{condition}{Condition}[section]
%\newtheorem{exercise}{Ex.}[section]
%\newtheorem{fact}{Fact}[section]


\usepackage[T1]{fontenc}
\newtheorem{theorem}{Theorem}
\newtheorem{acknowledgement}{Acknowledgement}
\newtheorem{assumption}{Assumption}
\newtheorem{corollary}{Corollary}
\newtheorem{criterion}{Criterion}
\newtheorem{definition}{Definition}
\newtheorem{example}{Example}
\newtheorem{lemma}{Lemma}
\newtheorem{proposition}{Proposition}
\newtheorem{remark}{Remark}
\newtheorem{hypothesis}{Hypothesis}
\newtheorem{observation}{Observation}
\newenvironment{proof}[1][Proof]{\noindent\textbf{#1.} }{\ \rule{0.5em}{0.5em}}
%\input{tcilatex}

\makeatletter
\def\title@font{\Large\bfseries}
\let\ltx@maketitle\@maketitle
\def\@maketitle{\bgroup%
	\let\ltx@title\@title%
	\def\@title{\resizebox{\textwidth}{!}{%
			\mbox{\title@font\ltx@title}%
	}}%
	\ltx@maketitle%
	\egroup}
\makeatother
\usepackage{setspace}
\usepackage{amsmath}
\onehalfspacing
\newenvironment{p_enumerate}{
	\begin{enumerate}
		\setlength{\itemsep}{1pt}
		\setlength{\parskip}{0pt}
		\setlength{\parsep}{0pt}
	}{\end{enumerate}}
\sectionfont{\centering\mdseries\scshape\bfseries}
\subsectionfont{\raggedright\mdseries\scshape\bfseries}
\subsubsectionfont{\flushleft\mdseries\itshape\bfseries}
\makeatletter
\def\@seccntformat#1{\csname the#1\endcsname.\quad}
\makeatother
\def\signed #1{{\leavevmode\unskip\nobreak\hfil\penalty50\hskip2em
		\hbox{}\nobreak\hfil(#1)%
		\parfillskip=0pt \finalhyphendemerits=0 \endgraf}}
\newsavebox\mybox
\newenvironment{aquote}[1]
{\savebox\mybox{#1}\begin{quote}}
	{\signed{\usebox\mybox}\end{quote}}

\pdfminorversion=4

\makeatletter
\newcommand{\changeoperator}[1]{%
	\csletcs{#1@saved}{#1@}%
	\csdef{#1@}{\changed@operator{#1}}%
}
\newcommand{\changed@operator}[1]{%
	\mathop{%
		\mathchoice{\textstyle\csuse{#1@saved}}
		{\csuse{#1@saved}}
		{\csuse{#1@saved}}
		{\csuse{#1@saved}}%
	}%
}
\makeatother

\changeoperator{sum}
\changeoperator{
	prod}

\begin{document}
	
	
	
	
	
	
	\vspace{-0.5ex}
	
	
	
	
	
	
	
	
	\newpage{}
	
	
	
	
	
	
	\title{{A Smorgasbord of Expectation Shocks
			}}
			
			
			
			%\date{This version: February, 2016\\
				%{First version: December, 2015}}
			\date{This version: November 2022}%\\This version: February, 2017}
		
		
		\author{Sonan Memon\footnote{Research Fellow, PIDE, Islamabad. \texttt{smemon@pide.org.pk}}}
		
		
		
		\newpage{}
		
		\maketitle
		\vspace{-2ex}
		
		
		
		
		
		
		
		
		
		\begin{center}
			\line(1,0){470}
		\end{center}
		\begin{spacing}{1.1}
			\vspace{-3ex}
			\begin{abstract}
				\noindent 
				I study a smorgasbord of different expectation shocks in two kinds of macroeconomic models. I present impulse response results for exogenous, temporary expectation shocks lasting for one period only \textit{or} 4 periods, permanent exogenous shocks (long run shock) and a series of multiple positive and negative, temporary exogenous shocks within a long period. As a baseline, I use a simple, aggregate demand and supply framework with adaptive expectations. Later, I extend my results by using a New Keynesian model with various choices of parameters, allowing for a richer analysis. The results indicate the centrality of expectation formation process in driving the shock reactions and propagation\footnote{The replication code of this paper with R, Julia and MATLAB code is available on my GitHub page: {\color{blue}{https://github.com/sonanmemon}}}.
				%I evaluate the forecasting potential of my method through VAR (Vector Autoregressive Models) models and forecast error variance decompositions (FEVD).
			\end{abstract}
		\end{spacing}
		\textbf{Keywords:} Smorgasbord of Inflation Expectation Shocks. Temporary, Permanent and Sequence of Temporary Expectation Shocks. Monetary Policy and Inflation Expectations. AD and AS Model. Expectation Shocks in New Keynesian Models.  {}\\
		\textbf{JEL Classification}: E00, E12, E30, E32, E40, E50, E52, E70, E71, D84.
		%\textbf{JEL Classifications:}
		\\
		\begin{center}
			\vspace{-8ex}
			\line(1,0){470}
		\end{center}
		\pagenumbering{arabic}
		\baselineskip=18pt 
		
		\newpage{}
		
		\begin{figure}[H]
			\begin{center}
				\includegraphics[width=0.4\linewidth]{pidelogo.jpg}		
				\caption*{}
			\end{center}
		\end{figure}
		
		\vspace{-8ex}
		
		
		
		\tableofcontents
		
		\newpage{}
		
		\vspace{-8ex}
		
		
		
		
		\section{Motivation}
	
		
	
		
		There is a large and growing literature in macroeconomics which attributes business cycle fluctuations to expectations, especially in light of the Great Recession, which did not seem to be driven by extremely unfavorable fundamentals. Many economists now recognize an enlarged role for beliefs in the narrative of business cycles (see for example \cite{veldkamp2019tail}, \cite{gennaioli2020crisis}). Classic studies such as those of \cite{minsky1977financial}, \cite{kindleberger1978manias} and more recently \cite{reinhart2009time} argue that the failure of investors to accurately assess risks is a common thread of many of these episodes. \cite{rajan2006has} and \cite{taleb2007black} stressed the dangers from low probability risks to financial stability due to subprime mortgages.
		
		 For instance, in October 2017, the University of Chicago surveyed
		 a panel of leading economists in the United States and Europe on the importance
		 of various factors contributing to the 2008 Global Financial
		 Crisis. The number one contributing factor among the panelists was the
		 ``flawed financial sector'' in terms of regulation and supervision. But the
		 number- two factor among the twelve considered, ranking just below
		 the first in estimated importance, was underestimation of risks from
		 financial engineering. The experts seem to agree that the fragility of a
		 highly leveraged financial system exposed to major housing risk was
		 not fully appreciated in the period leading to the crisis. Many economists increasingly recognize that the Lehman bankruptcy and the fire sales during 2008 revealed that investors and policymakers learned that the financial system was more fragile and interdependent than they previously thought \cite{gennaioli2020crisis}.
		 
		 
		 If the output over-expansion is fueled by excessive credit growth, as suggested by recent historical evidence \cite{schularick2012credit}, \cite{mian2017household}\footnote{\cite{mian2017household} provide global and historical evidence that rising household debt predicts recessions and intensity of recessions is related to prior debt expansion in household sector.}, then eventual recognition of tail risks and overheating in financial markets paves the way for a \textit{Minsky Moment} \cite{minsky1977financial}. For instance, \cite{bordalo2018diagnostic} build a micro-founded and behavioral model of expectations called \textit{diagnostic expectations} and credit cycles in which beliefs overreact to incoming news because of the representative heuristic. This creates excessive optimism when credit spreads are low, during booms and also exaggeration of subsequent reversal when good news inflow slows down, leading to endogenous cycles in absence of change in fundamentals, engendering a recession endogenously.
		 
		 
		 Much of this work indicates that there are errors in expectations over the course of the business cycle. This has led to the trend of data collection by various central banks in the world such as the Federal Reserve in USA and even the State Bank of Pakistan, on expectations through survey data. Increasingly, such data is considered a valid and extremely useful source of information for economic research. We have learned that expectations in financial markets tend to be extrapolative rather than rational and this basic feature needs to be integrated into economic analysis.
		 
		 
		 
		
		
		
		
		In this work, my focus is on \textit{modeling expectation shocks} in a simple aggregate demand and supply model with adaptive expectations as a baseline, followed by an extension into a New Keynesian model with a combination of forward looking and adaptive expectations. I study a variety of expectation shocks such as temporary shocks, lasting for one period versus those which lost for four periods, permanent shocks and a series of repeated temporary shocks. In doing so, I analyze the responses of inflation, output, nominal and real interest rates in reaction to various expectation shocks. Lastly, I use some stylized data from Pakistan and analyze the impulse responses of key macroeconomic variables to such expectation shocks, in a developing economy with lower levels of financial access and high poverty.
		
		
		
		
		
		
		
		
		
		
		\section{Stylized Facts} 
		
		
		
		 There is a structure, pattern, regularity and relative coherence in the manner in which consumer expectations evolve over the business cycle, especially when one examines cross-sectional heterogeneity. Certain demographic groups have consistently more pessimistic and inaccurate expectations such as women, ethnic minorities, lower socioeconomic groups and young people (see for instance \cite{madeira2015heterogeneous, curtin19}). There is also an average pessimism bias across all demographic groups because of asymmetric recall of negative news in the elicited expectations, relative to estimates of rational expectations \cite{curtin19}, \cite{bhandari2019survey}. The volatility of consumer sentiment over the business cycle also varies across groups with higher socio-economic groups showing more volatility \cite{curtin19}. Meanwhile, the time series co-movements across demographic groups are very high. 
		 
		 Moreover, the literature has established that consumer sentiment indices regularly predict recessions, though not by a long horizon. In fact, the forward looking, informative and leading indicator nature of consumer sentiment data is precisely the reason why the University of Michigan survey and similar surveys have become globally popular among central banks and policy makers. This evidence suggests that while ``autonomous'' components of consumer sentiment such as those driven by instruments are needed for econometric identification of plausibly exogenous variation, there is also an important \textit{systematic} and \textit{endogenous} component to these sentiments which is responding to, predicting and causing significant developments in the real economy. For instance, the sentiments can influence search intensity in labor markets, consumer durable goods purchases and so on. In fact, there is evidence that household expectations are predictive of economic and financial behavior \cite{armantier2015inflation, armona2018home} and high volatility in consumer durable goods purchases over the business cycle has been often attributed in the literature to consumer sentiment fluctuations (see for instance \cite{katona1960powerful, mishkin1978consumer}).
		 
		 
		 \subsection{Data From Pakistan}
		
		
	
		
		
		The SBP conducts various surveys, including Consumer Confidence Survey (CCS) and Business Confidence Survey (BCS) after every two months. CCS is the telephonic survey of households that are selected randomly across the country, and provides the information on ``what people are thinking'' about the economy.  BCS is the telephonic survey of firms and provides the information on what firms are thinking about the business conditions in the country. 
		
		%The State Bank also undertakes the Bank Lending Survey (BLS) and Systemic Risk Survey (SRS). BLS is an online web based survey of lenders to obtain current and expected credit conditions and the major factors affecting those conditions. SRS is a email based survey of market participants and experts about various existing and emerging/potential risks and their confidence in the stability of the financial system. 
		
		In the graphs presented next, I use data from State Bank of Pakistan (SBP) on consumer confidence in Pakistan during 2012 to 2022.  Firstly, in Figure 1, I plot the evolution of three consumer confidence indices for Pakistan, at a bi-monthly (i.e six times in one year or once every two months) frequency from January 2012 to September 2022. It is evident that all three indices: overall consumer confidence index (CCI), current economic conditions index (CEC) and expected economic conditions index (EEC) co-move with each other. However, since 2018 some variation is noticeable, with expected economic conditions being the most optimistic, perception of current economic conditions being at the lowest level of optimism and consumer confidence index lying somewhere in the middle. The data also reveals that the recent inflation crisis in Pakistan led to a sharp reduction in consumer confidence in early 2022, which has only mildly recovered by September 2022.
	
		
		\begin{figure}[H]
			\centering
			\scalebox{0.8}{\input{CC_Pakistan.tex}}
			\caption[]{Bi-Monthly Consumer Confidence Indices (2012-2022)}
		\end{figure}
	
		
	
	Meanwhile in Figure 2, I have plotted the bi-monthly inflation expectations index which includes categories such as energy products, food and non-food inflation and lastly daily use items. It is evident that when consumer confidence and expectations regarding economic conditions were becoming more optimistic over time during 2012 to 2018, six month ahead inflation expectations were also falling. During 2018 to 2021, when consumer confidence fell, it was coterminous with a rise is short run, inflation expectations. Across various items, the inflation expectations were fairly similar but they rose dramatically for daily use items during 2016 to 2018 relative to the other categories. Whereas, energy items tend to be associated with lower average, inflation expectations relative to all other items, especially daily use products and these expectations are highly volatile, especially driven by frequent bouts of dramatically lower inflation expectations relative to other groups.

		
		
		
		
		
		\begin{figure}[H]
			\centering
			\scalebox{0.8}{\input{InflationExpectations.tex}}
			\caption[]{Bi-Monthly Inflation Expectations (2012-2021)}
		\end{figure}
	
	
	In Figure 3, I provide evidence on cross-correlations between quarterly expected economic conditions index and quarterly GDP data for Pakistan during 2012 to 2021, based on SBP's (State Bank of Pakistan) data i.e $Corr(EEC_{x-t}, GDP_{x})$, where $t \in (-10,10)$. The results below, along with 95\% confidence intervals reveal that while increase in past levels of expected economic conditions are positively and significantly correlated with future real GDP growth rates at various horizons, especially 5 or less quarters. Meanwhile, changes in current real GDP growth are not significantly correlated with future expected economic conditions\footnote{In the appendix, I also present graphs which help visualize the leading role of expected economic conditions, relative to real GDP growth.}.
		
		
	\begin{figure}[H]
		\centering
		\scalebox{0.8}{\input{CCF_EEC_GDP2018.tex}}
		\caption[]{Cross-Correlation Function for Expected Economic Conditions and GDP}
	\end{figure}


The SBP Pakistan has also measured business confidence index since the end of 2017. In Figure 4, I plot the expected economic conditions (EEC) index, expected exchange rate relative to US dollar (EER) and expected inflation index (EI) based on business sector surveys from November 2017 to September 2022.

\begin{figure}[H]
	\centering
	\scalebox{0.8}{\input{BCI.tex}}
	\caption[]{Bi-Monthly Business Confidence Indices (2017-2022)}
\end{figure}


		
		
		\section{Aggregate Demand and Aggregate Supply Model}
		
		 I begin with a simple, backward looking, textbook aggregate demand and supply model as in \cite{abel2017macroeconomics} with standard demand equation, a Fisher equation representing the relationship between real and nominal interest rates, a Philip's curve, adaptive expectations and a monetary policy rule or Taylor rule. The expectation formation process is adaptive, which implies expectation of inflation si merely extrapolating from the past inflation in addition to an error term, which will be the source of shocks.
		
		\subsection{Building Blocks}
		
		Output Equation/Demand for Goods and Services: 
		\begin{equation}
			Y_{t} = \bar{Y} - \alpha (r_{t} - \rho) + \epsilon_{t}, \; \alpha > 0
		\end{equation}
	
	
	Fisher Equation: 
	\begin{equation}
		r_{t} = i_{t} - \mathbb{E}_{t}\{\pi_{t+1}\}
	\end{equation}
		
		Philip's Curve: 
		\begin{equation}
			\pi_{t} = i_{t} - \mathbb{E}_{t-1}\{\pi_{t}\} + \phi(Y_{t} - \bar{Y}) + v_{t}, \; \phi > 0
		\end{equation}
		
		Adaptive Expectations:
		\begin{equation}
			\mathbb{E}_{t}\{\pi_{t+1}\} = \pi_{t} + \eta_{t}, \; \forall t
		\end{equation}
	
	Monetary Policy Rule:
	\begin{equation}
		i_{t} = \pi_{t} + \rho + \theta_{\pi} (\pi_{t} - \pi^{*}) + \theta_{Y} (Y_{t} - \bar{Y}), \; \theta_{\pi}, \theta_{Y} > 0
	\end{equation}


%\begin{equation*}
%	\begin{split}
%	Y_{t} = \bar{Y} - \alpha (r_{t} - \rho) + \epsilon_{t}, \; \alpha > 0 \\
%	r_{t} = i_{t} - \mathbb{E}_{t}\{\pi_{t+1}\} \\
%	\pi_{t} = i_{t} - \mathbb{E}_{t-1}\{\pi_{t}\} + \phi(Y_{t} - \bar{Y}) + v_{t}, \; \phi > 0 \\
%	\mathbb{E}_{t}\{\pi_{t+1}\} = \pi_{t}, \; \forall t \\
%	i_{t} = \pi_{t} + \rho + \theta_{\pi} (\pi_{t} - \pi^{*}) + \theta_{Y} (Y_{t} - \bar{Y}), \; \theta_{\pi}, \theta_{Y} > 0
%	\end{split}
%\end{equation*}










	
		\subsection{Long Run Equilibrium}
		
		The long run equilibrium, which is equivalent to the steady state in this simple model satisfies the following conditions. After responding to a temporary shock, all variables would eventually converge back to this equilibrium.
		
		
		\begin{equation*}
			\begin{split}
			Y_{t} = \bar{Y} \\
			r_{t} = \rho \\
			\pi_{t} = \pi^{*} \\
			\mathbb{E}_{t}\{\pi_{t+1}\} = \pi^{*} \\
			i_{t} = \rho + \pi^{*}
			\end{split}
		\end{equation*}
	
	\subsection{Parameters}
	
	The steady state output i.e $\bar{Y} = 50$, steady state inflation i.e $\pi^{*} = 2$ or 2\%, the baseline responsiveness to inflation $\phi_{\pi} = 1$ in the taylor rule and responsiveness to output is $\phi_{Y} = 0.3$. The natural rate of interest i.e $\rho = 2\%$ and the responsiveness of demand to $r_{t}$ (real interest rates) is measured by $\alpha$.
	
	
	\begin{center}
		\Taccount{Model Parameters}{$\bar{Y} = 50$&$\pi^{*} = 2$\\$\rho = 2$&$\alpha=1$\\$\theta_{\pi} = 1$&$\theta_{Y} = 0.3$\\$\phi = 0.6$}
	\end{center}
		
		\subsection{Dynamic AS and Dynamic AD Equations}
		
		In this section, I derive the two central equations of this aggregate demand and supply model i.e the dynamic AD and dynamic AS equations. 
		
		The dynamic AS curve is displayed in equation 6 below:
			\begin{equation}
			\pi_{t} = \pi_{t-1} + \eta_{t-1} + \phi (Y_{t} - \bar{Y}) + v_{t}
		\end{equation}
	
	The dynamic AD curve is displayed in equation 7 below:
		\begin{equation}
		Y_{t} = \bar{Y} - \frac{\alpha \theta_{\pi}}{1 + \alpha \theta_{Y}}(\pi_{t} - \pi^{*}) + \frac{1}{1 + \alpha \theta_{Y}} \epsilon_{t} + \frac{\alpha}{1 + \alpha \theta_{Y}} \eta_{t}
	\end{equation}

In equilibrium, aggregate demand equals aggregate supply, which implies that:
\begin{equation*}
\pi_{t} = \pi_{t-1} + \eta_{t-1} + \phi\bigg(\bar{Y} - \frac{\alpha \theta_{\pi}}{1 + \alpha \theta_{Y}}(\pi_{t} - \pi^{*}) + \frac{1}{1 + \alpha \theta_{Y}} \epsilon_{t} + \frac{\alpha}{1 + \alpha \theta_{Y}} \eta_{t} - \bar{Y}\bigg) + v_{t}
\end{equation*}

Some further simplification yields:

\begin{equation*}
	\pi_{t}\bigg(1 +  \frac{\phi \times \alpha \times \theta_{\pi}}{1 + \alpha \theta_{Y}}\bigg)  = \pi_{t-1} + \eta_{t-1} + \phi\bigg(\frac{\alpha \theta_{\pi}}{1 + \alpha \theta_{Y}} \times \pi^{*} + \frac{1}{1 + \alpha \theta_{Y}} \epsilon_{t} + \frac{\alpha}{1 + \alpha \theta_{Y}} \eta_{t} \bigg) + v_{t}
\end{equation*}

Using some further notation for the purposes of simplification and assuming that $v_{t} = 0$ (assuming no supply shocks), I derive the following equations ($8$ and $9$) for inflation and output in equilibrium. These equations can be solved for equilibrium levels of $\pi_{t}$ and $Y_{t}$ in any period, given the shocks, exogenous parameters (defined in last section) and past values\footnote{This is a backward looking model.} of $\pi_{t-1}$ and $\eta_{t-1}$. Thus, one can compute the impulse responses for any forward horizon, given any initial shock to either $\eta_{t}$ (expectation shock) or $\epsilon_{t}$ (demand shock). 

For instance, let's assume that we were in the long run equilibrium (i.e $\pi_{t-1} = \pi^{*} = 2\%$, $\bar{Y} = 50$, $i^{*} = 4\%$ and $r^{*} = 2\%$) before a positive, exogenous and one period (temporary) expectation shock i.e $\eta_{t} = 1$ hits the economy during period 1. In this case, can compute the impulse responses for inflation and output (using 8 and 9), before computing them for nominal and real interest rates (using equations 10 and 11 after we have solved for $\pi_{t}$ and $Y_{t}$). Figure 1 of section 3 below depicts the impulse responses (50 periods) for exactly such a one period expectation shock.

\begin{equation}
	\pi_{t}  = \frac{\pi_{t-1} + \eta_{t-1} + \gamma \times \pi^{*} + \theta \times \epsilon_{t} +  \beta \eta_{t}}{\zeta}
\end{equation}

\begin{equation}
	Y_{t}  = \bar{Y} - \frac{\gamma}{\phi} \bigg(\pi_{t} - \pi^{*}\bigg) + \frac{\theta}{\phi} \epsilon_{t} + \frac{\beta}{\phi} \eta_{t}
\end{equation}


	

\begin{equation}
	i_{t} = \pi_{t} + \rho + \theta_{\pi} (\pi_{t} - \pi^{*}) + \theta_{Y} (Y_{t} - \bar{Y}), \; \theta_{\pi}, \theta_{Y} > 0
\end{equation}

\begin{equation}
	r_{t} = i_{t} - (\pi_{t} + \eta_{t})
\end{equation}

Note that $\zeta = \bigg(1 +  \frac{\phi \times \alpha \times \theta_{\pi}}{1 + \alpha \theta_{Y}}\bigg)$, $\gamma = \bigg(\frac{\alpha \times \phi \times \theta_{\pi}}{1 + \alpha \theta_{Y}}\bigg)$, $\theta = \bigg(\frac{\phi}{1 + \alpha \theta_{Y}}\bigg)$, $\beta = \bigg(\frac{\phi \times \alpha}{1 + \alpha \theta_{Y}}\bigg)$.














%\Taccount{Expenses}{Debits&Credits\\Increase&Decrease\\Normal Balance}\quad
%\Taccount{Owner's Drawing}{Debits&Credits\\Increase&Decrease\\Normal Balance}
%\bigskip
%\Taccount{Liabilities}{Debits&Credits\\Decrease&Increase\\&Normal Balance}\quad
%\Taccount{Revenues}{Debits&Credits\\Decrease&Increase\\&Normal Balance}\quad
%\Taccount{Owner's Capital}{Debits&Credits\\Decrease&Increase\\&Normal Balance}
	
		
		
	
		
		\section{Impulse Responses}
		
		
		
		
		
	All of the graphs in this section, show responses to expectation shocks i.e various type of shocks to $\eta_{t}$.
	
	In Figure 5, I perturb the system with a one period, temporary shock to expectations and observe the response of inflation, output, real and nominal interest rates. 
		
		
		
		
		
		
		\begin{figure}[H]
				\includegraphics[width=0.5\linewidth]{inflation_expectationshock.pdf}
				\hfill
				\includegraphics[width=0.5\linewidth]{output_expectationshock.pdf}
				\hfill
				\begin{Center}
				\includegraphics[width=0.5\linewidth]{interestrates_expectationshock.pdf}
				\end{Center}
				\caption{Impulse Responses For 1 Period Shock}
		\end{figure}
	
	In response to a 4 period shock to expectations, we observe the following
		
		
			\begin{figure}[H]
			\includegraphics[width=0.5\linewidth]{inflation_expectationshock4period.pdf}
			\hfill
			\includegraphics[width=0.5\linewidth]{output_expectationshock4period.pdf}
			\hfill
			\begin{Center}
				\includegraphics[width=0.5\linewidth]{interestrates_expectationshock4period.pdf}
			\end{Center}
			\caption{Impulse Responses For 4 Period Shock}
		\end{figure}
	
	
	\begin{figure}[H]
		\includegraphics[width=0.5\linewidth]{inflation_expectationshock4periodoutputpref.pdf}
		\hfill
		\includegraphics[width=0.5\linewidth]{output_expectationshock4periodoutputpref.pdf}
		\hfill
		\begin{Center}
			\includegraphics[width=0.5\linewidth]{interestrates_expectationshock4periodoutputpref.pdf}
		\end{Center}
		\caption{Impulse Responses For 4 Period Shock and Output Preference}
	\end{figure}


	
The following figure displays responses to a permanent shock to expectations.
	
	\begin{figure}[H]
		\includegraphics[width=0.5\linewidth]{inflation_expectationshock50period.pdf}
		\hfill
		\includegraphics[width=0.5\linewidth]{output_expectationshock50period.pdf}
		\hfill
		\begin{Center}
			\includegraphics[width=0.5\linewidth]{interestrates_expectationshock50period.pdf}
		\end{Center}
		\caption{Impulse Responses For Permanent Shock}
	\end{figure}


Lastly, I analyze a sequence of 3 temporary shocks, each lasting for four periods, where the second shock is negative and the remaining two are positive expectation shocks.


	\begin{figure}[H]
	\includegraphics[width=0.5\linewidth]{inflation_seriesofexpectationshocks.pdf}
	\hfill
	\includegraphics[width=0.5\linewidth]{output_seriesofexpectationshocks.pdf}
	\hfill
	\begin{Center}
		\includegraphics[width=0.5\linewidth]{interestrates_seriesofexpectationshocks.pdf}
	\end{Center}
	\caption{Impulse Responses For Series of 3 Temporary (4 Period Each) Shocks}
\end{figure}
		
		
		\newpage
		
		
			
		
		\section{New Keynesian Model}
		
		\subsection{Framework}
		
		I begin by using a simple, stylized, three equation New Keynesian model as developed in \cite{gali2015book}. The following three (12 to 14) equation represent thee NKPC (New Keynesian Philip's Curve), Output Gap Equation and Taylor rule.
		
	New Keynesian Philip's curve: 
	
	\begin{equation}
		\pi_{t} = \beta \mathbb{E}_{t}\{\pi_{t+1}\} + \kappa \: \widetilde{y}_{t}
	\end{equation}

Output Gap Equation:

	\begin{equation}
	\widetilde{y_{t}} = -\frac{1}{\sigma} \big(i_{t} - \mathbb{E}_{t}\{\pi_{t+1}\} - r_{t}^{n}\big) + \mathbb{E}_{t}\{\widetilde{y_{t+1}}\}
\end{equation}

Interest Rate Rule (Taylor Rule):

\begin{equation}
	i_{t} = \rho + \phi_{\pi} \pi_{t} + \phi_{y} \widetilde{y_{t}} + v_{t} 
\end{equation}


The three equations stated above can be combined and represented as a system of difference equations which has the following representation:


%\begin{tiny}
	\begin{equation}
		\begin{bmatrix}
			\widetilde{y_{t}}  \\
			\pi_{t} \\
		\end{bmatrix} = {\bf{A}}_{T} \begin{bmatrix}
		\mathbb{E}_{t}\{\widetilde{y_{t+1}}\}  \\
		\mathbb{E}_{t}\{{\pi_{t+1}}\} \end{bmatrix} + {\bf{B}}_{T} u_{t} \\
	\end{equation}
%\end{tiny}


%\begin{tiny}
In the above system, ${\bf{A}}_{T}$, ${\bf{B}}_{T}$, $\Omega$ and $u_{t}$ are defined as follows:

	\begin{equation*}
		{\bf{A}}_{T} \equiv \Omega 
		\begin{bmatrix}
			\sigma & 1 - \beta \phi_{\pi}  \\
			\sigma \times \kappa & \kappa + \beta (\sigma + \phi_{y}) \\
		\end{bmatrix}, \:
		{\bf{B}}_{T} \equiv \Omega \begin{bmatrix}
			1 \\
			\kappa \\
		\end{bmatrix}, \: \Omega \equiv \frac{1}{\sigma + \phi_{y} + \kappa \phi_{\pi}}
	\end{equation*}
\begin{equation*}
and \: u_{t} \equiv \psi_{ya}(\phi_{y} + \sigma (1 - \rho_{a}))a_{t} + (1 - \rho_{z})z_{t} - v_{t}
\end{equation*}

Note that the natural rate of interest $r_{t}^{n}$ can be defined as: $r_{t}^{n} = \rho - \sigma (1 - \rho_{\alpha}) \psi_{ya} a_{t} + (1 - \rho_{z}) z_{t}$, where $z_{t}$ is the discount rate shock (shock to consumer utility or demand), $\alpha_{t}$ is the technology shock (supply shock or production shock) and $v_{t}$ is the monetary policy shock (i.e a deviation from the monetary policy rule). Moreover, $\psi_{ya} = \frac{1 + \varphi}{\sigma (1 - \alpha) + \varphi + \alpha}$ and $\widetilde{y_{t}} = y_{t} - y_{t}^{n}$, so that the output gap i.e $\widetilde{y_{t}}$ is the deviation of output from its natural rate $ y_{t}^{n}$. All the three exogenous shocks are represented as AR(1)\footnote{autoregressive processes of order 1.} processes i.e $z_{t} = \rho_{z} z_{t-1} + \epsilon_{t}^{z}$, $v_{t} = \rho_{v} v_{t-1} + \epsilon_{t}^{v}$ and $a_{t} = \rho_{a} a_{t-1} + \epsilon_{t}^{a}$. For a display of all key model equations, refer to the appendix, section 7.2 and for an even more detailed exposition on the baseline, New Keynesian model refer to \cite{gali2015book}\footnote{Chapter 3, Basic New Keynesian (henceforth NK) Model in \cite{gali2015book}.}, which also includes the conditions on parameters needed for a unique, local solution to this model\footnote{Assuming that $\phi_{\pi}$ and $\phi_{y}$ are non-negative coefficients, it has been shown by \cite{bullard2002learning} that the necessary and sufficient condition for a unique, local equilibrium is $\kappa (\phi_{\pi} - 1) + (1 - \beta) \phi_{y} > 0$}. 

The baseline parameterization is displayed in the following table and is consistent with the literature. The elasticity of intertemporal substitution is set to $\sigma = 1$and discount factor i.e $\beta = 0.99$.

	\begin{center}
	\Taccount{NK Model Parameters}{$\sigma = 1$&$\varphi = 5$\\$\phi_{\pi} = 0.5$&$\phi_{y}=0.125$\\$\theta = 0.75$&$\rho_{\nu} = 0.5$\\$\rho_{z} = 0.6$&$\eta=3.77$\\$\rho_{a} = 0.9$&$\beta=0.99$\\$\alpha=0.25$&$\epsilon=9$}
\end{center}


\subsection{Impulse Responses}



The following figures plot the dynamic responses of various variables to a temporary, negative shock to discount rates or discount factor shock i.e negative shock to $\epsilon_{t}^{z}$ or a decrease in $z_{t}$\footnote{The plots were generated using Dynare and code was motivated by the code, produced by Dr. Johannes Pfeifer (see \textcolor{blue}{https://github.com/JohannesPfeifer}).}. This shock can be interpreted as causing a \textit{reduction} in the weight that households assign to current utility, relative to future utility. In the following diagrams, ``ann'' refers to annualized, pi refers to inflation ($\pi$ in the above equations), $y_{gap}$ refers to output gap i.e $\widetilde{y_{t}}$ in the model above, $p$ refers to price levels, $i$ refers to nominal interest rates, $r$ refers to real interest rates, $m$ refers to money supply, $n$ refers to hours worked and $w$ refers to real wages.

Figure 14 plots the impulse responses to a temporary, negative discount rate shock with forward looking expectations. The impulses reveals that if we work with the baseline 3 equation New Keynesian model, a negative shock to discount rates makes the output gap negative, as the output deviates downwards from steady state and so do hours worked, real wages and output. Both the real and nominal interest rates at annualized levels fall in reaction to the shock while annualized inflation falls in the short run. Meanwhile, the price levels depreciate and display a persistent effect of the shock, which lasts for several years, stabilizing close to negative 20\% after 6 periods. Lastly, after an initial appreciation in nominal money supply we observe a long term degrowth in money supply, which is close to 20\%.








	

	
	
	\begin{figure}[H]
		\begin{center}
			\includegraphics[width=1\linewidth]{chap3_IRF_eps_z1.eps}
		\end{center}
		\caption{Dynamic Responses to Stochastic Discount Rate Shock with Forward Looking $\mathbb{E}$}
	\end{figure}


Meanwhile, Figure 15 plots the responses to a negative, discount rate shock with backward looking, adaptive expectations in an otherwise New Keynesian model. In this case, both the output and output gap display an initial downward trajectory, followed by a brief and mild expansion after 4 periods, which lasts for roughly 7 periods and slowly tapers off. Similar dynamics are displayed by hours worked $n$ and real wages. The price levels display a continuous and monotonic fall after the shock period. However, annualized inflation slowly recovers toward the steady state after the initial downward shock. Both nominal and real interest rates depreciate in response to the shock, followed by a gradual recovery toward steady state but the nominal interest rates is more rigid in its reaction to the shock after initial contraction due to backward looking expectations. Lastly, nominal money supply appreciates in reaction to the discount rate shock, before displaying hump shaped persistence for some periods and finally after 4 periods begins its depreciation toward the degrowth, slowing down by 40\% relative to steady state by the 14th period.


\begin{figure}[H]
	\begin{center}
		\includegraphics[width=1\linewidth]{chap3_IRF_eps_z1-backwardlookingexp.eps}
	\end{center}
	\caption{Dynamic Responses to Stochastic Discount Rate Shock with Backward Looking $\mathbb{E}$}
\end{figure}



\begin{figure}[H]
	\begin{center}
		\includegraphics[width=1\linewidth]{chap3_IRF_tauzaniticipated8aheadshock-forwardlooking.eps}
	\end{center}
	\caption{Anticipated (8 Period Ahead) Discount Rate Shock with Forward Looking $\mathbb{E}$}
\end{figure}


\begin{figure}[H]
	\begin{center}
		\includegraphics[width=1\linewidth]{chap3_IRF_tauzaniticipated8aheadshock-backwardlooking.eps}
	\end{center}
	\caption{Anticipated (8 Period Ahead) Discount Rate Shock with Backward Looking $\mathbb{E}$}
\end{figure}










		
		\newpage
		
		\section{Conclusion}
		
		
		
		
		\newpage
		
		
		
		
		
		\section{Appendix}
		
		\subsection{Consumer Confidence and Real GDP}
		
		\begin{figure}[H]
			\centering
			\scalebox{0.8}{\input{GrowthinCC_GDP.tex}}
			\caption[]{Growth Rate (\%) in Consumer Confidence Indices and Real GDP}
		\end{figure}
	
	\subsection{NK Model}
	
	\begin{equation*}
		\Omega = \frac{1-{{\alpha}}}{1-{{\alpha}}+{{\alpha}}\, {{\epsilon}}}
	\end{equation*}

\begin{equation*}
	\psi\_n\_ya = \frac{1+{{\varphi}}}{{{\alpha}}+{{\varphi}}+\left(1-{{\alpha}}\right)\, {{\sigma}}}
\end{equation*}
\begin{equation*}
	\lambda = \frac{\left(1-{{\theta}}\right)\, \left(1-{{\theta}}\, {{\beta}}\right)}{{{\theta}}}\, {\Omega}
\end{equation*}
\begin{equation*}
	\kappa = {\lambda}\, \left({{\sigma}}+\frac{{{\alpha}}+{{\varphi}}}{1-{{\alpha}}}\right)
\end{equation*}
% Equation 1
\begin{equation*}
	{{\pi}}={{\beta}}\, {{\pi}}+{\kappa}\, {{\tilde y}}
\end{equation*}
% Equation 2
\begin{equation*}
	{{\tilde y}}={{\tilde y}}+\frac{\left(-1\right)}{{{\sigma}}}\, \left({{i}}-{{\pi}}-{{r^{nat}}}\right)
\end{equation*}
% Equation 3
\begin{equation*}
	{{i}}={{\pi}}\, {{\phi_{\pi}}}+{{\phi_{y}}}\, {{\hat y}}+{{\nu}}
\end{equation*}
% Equation 4
\begin{equation*}
	{{r^{nat}}}=\left(-{{\sigma}}\right)\, {\psi\_n\_ya}\, \left(1-{{\rho_a}}\right)\, {{a}}+\left(1-{{\rho_{z}}}\right)\, {{z}}
\end{equation*}
% Equation 5
\begin{equation*}
	{{r^r}}={{i}}-{{\pi}}
\end{equation*}
% Equation 6
\begin{equation*}
	{{y^{nat}}}={\psi\_n\_ya}\, {{a}}
\end{equation*}
% Equation 7
\begin{equation*}
	{{\tilde y}}={{y}}-{{y^{nat}}}
\end{equation*}
% Equation 8
\begin{equation*}
	{{\nu}}={{\nu}}\, {{\rho_{\nu}}}+{{\varepsilon_\nu}}
\end{equation*}
% Equation 9
\begin{equation*}
	{{a}}={{\rho_a}}\, {{a}}+{{\varepsilon_a}}
\end{equation*}
% Equation 10
\begin{equation*}
	{{y}}={{a}}+\left(1-{{\alpha}}\right)\, {{n}}
\end{equation*}
% Equation 11
\begin{equation*}
	{{z}}={{\rho_{z}}}\, {{z}}-{{\varepsilon_z}}-{{\tau_z}}
\end{equation*}
% Equation 12
\begin{equation*}
	{{\Delta m}}={{\pi}}\, 4
\end{equation*}
% Equation 13
\begin{equation*}
	{{m-p}}={{y}}-{{i}}\, {{\eta}}
\end{equation*}
% Equation 14
\begin{equation*}
	{{i^{ann}}}={{i}}\, 4
\end{equation*}
% Equation 15
\begin{equation*}
	{{r^{r,ann}}}={{r^r}}\, 4
\end{equation*}
% Equation 16
\begin{equation*}
	{{r^{nat,ann}}}={{r^{nat}}}\, 4
\end{equation*}
% Equation 17
\begin{equation*}
	{{\pi^{ann}}}={{\pi}}\, 4
\end{equation*}
% Equation 18
\begin{equation*}
	{{\hat y}}={{y}}-({{y}})
\end{equation*}
% Equation 19
\begin{equation*}
	{{\pi}}=0
\end{equation*}
% Equation 20
\begin{equation*}
	{{y}}={{c}}
\end{equation*}
% Equation 21
\begin{equation*}
	{{w}}-{{p}}={{\sigma}}\, {{c}}+{{\varphi}}\, {{n}}
\end{equation*}
% Equation 22
\begin{equation*}
	{{\frac{w}{p}}}={{w}}-{{p}}
\end{equation*}
% Equation 23
\begin{equation*}
	{{m}}={{m-p}}+{{p}}
\end{equation*}
% Equation 24
\begin{equation*}
	{{\mu}}={{y}}\, \left(-\left({{\sigma}}+\frac{{{\alpha}}+{{\varphi}}}{1-{{\alpha}}}\right)\right)+{{a}}\, \frac{1+{{\varphi}}}{1-{{\alpha}}}
\end{equation*}
% Equation 25
\begin{equation*}
	{{\hat \mu}}={{\tilde y}}\, \left(-\left({{\sigma}}+\frac{{{\alpha}}+{{\varphi}}}{1-{{\alpha}}}\right)\right)
\end{equation*}
		
	
	
	\newpage
	%_________________ End of Main Matter_________________%
	%_________________ Reference Section _______________%
	\phantomsection % allows for correct link to Table of Contents
	\addcontentsline{toc}{section}{References} % Adds the line "References" to Table of contents
	\singlespacing
	%\bibliography{references0} % Uses the Bibtex-file mybibfile.bib
	\newpage
	\bibliographystyle{aer}
	\bibliography{references0}
	\clearpage
	%_________________ Space for Supplementary Material _______________%
	
	
	
	
	%\section*{Appendix}
	
	
	
	
	
\end{document}


